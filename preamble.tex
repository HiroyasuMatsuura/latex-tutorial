%
% Common preamble for all three parts.
%

\usepackage[english]{babel}
\usepackage{amsmath}
\usepackage{color}
\usepackage{minted}
\usepackage{hyperref}
\usepackage{multicol}
\usepackage{xcolor}
\usepackage{tabularx}
\usepackage{tikz}
\usetikzlibrary{arrows,intersections}
\usetikzlibrary{decorations.fractals}
\usepackage{tikz-cd}

\hypersetup{
    colorlinks=true,
    linkcolor=blue,
    filecolor=magenta,      
    urlcolor=cyan,
}
            
% Set Japanese font to Gothic
\usepackage[deluxe]{otf}
\renewcommand{\kanjifamilydefault}{\gtdefault}

% Set math font
\usefonttheme{professionalfonts}
\newcommand{\im}{\mathrm{i}}

% only inline todonotes work
\usepackage{xkeyval}
\usepackage[textsize=small]{todonotes}
\presetkeys{todonotes}{inline}{}

\usetikzlibrary{shapes,arrows,positioning,shadows}

% no nav buttons
\usenavigationsymbolstemplate{}

\newcommand{\bftt}[1]{\textbf{\texttt{#1}}}
\newcommand{\comment}[1]{{\color[HTML]{008080}\textit{\textbf{\texttt{#1}}}}}
\newcommand{\cmd}[1]{{\color[HTML]{008000}\bftt{#1}}}
\newcommand{\bs}{\char`\\}
\newcommand{\cmdbs}[1]{\cmd{\bs#1}}
\newcommand{\lcb}{\char '173}
\newcommand{\rcb}{\char '175}
\newcommand{\cmdbegin}[1]{\cmdbs{begin\lcb}\bftt{#1}\cmd{\rcb}}
\newcommand{\cmdend}[1]{\cmdbs{end\lcb}\bftt{#1}\cmd{\rcb}}

\newcommand{\wllogo}{\textbf{Overleaf}}

% this is where the example source files are loaded from
% do not include a trailing slash
\newcommand{\fileuri}{https://raw.github.com/jdleesmiller/latex-course/master/en}

\newcommand{\wlserver}{https://www.overleaf.com}
\newcommand{\wlnewdoc}[1]{\wlserver/docs?snip\_uri=\fileuri/#1\&splash=none}

\def\tikzname{Ti\emph{k}Z}

% from http://tex.stackexchange.com/questions/5226/keyboard-font-for-latex
\newcommand*\keystroke[1]{%
  \tikz[baseline=(key.base)]
    \node[%
      draw,
      fill=white,
      drop shadow={shadow xshift=0.25ex,shadow yshift=-0.25ex,fill=black,opacity=0.75},
      rectangle,
      rounded corners=2pt,
      inner sep=1pt,
      line width=0.5pt,
      font=\scriptsize\sffamily
    ](key) {#1\strut}
  ;
}
\newcommand{\keystrokebftt}[1]{\keystroke{\bftt{#1}}}


\let\olditemize\itemize
\renewcommand{\itemize}{
\olditemize
\setlength{\itemsep}{1.2pt}
\setlength{\parskip}{5pt}
\setlength{\parsep}{10pt}
}


% stolen from minted.dtx
\newenvironment{exampletwoup}
  {\VerbatimEnvironment
   \begin{VerbatimOut}{example.out}}
  {\end{VerbatimOut}
   \setlength{\parindent}{0pt}
   \fbox{\begin{tabular}{l|l}
   \begin{minipage}{0.55\linewidth}
     \inputminted[fontsize=\small,resetmargins]{latex}{example.out}
   \end{minipage} &
   \begin{minipage}{0.35\linewidth}
     \input{example.out}
   \end{minipage}
   \end{tabular}}}

\newenvironment{exampletwouptiny}
  {\VerbatimEnvironment
   \begin{VerbatimOut}{example.out}}
  {\end{VerbatimOut}
   \setlength{\parindent}{0pt}
   \fbox{\begin{tabular}{l|l}
   \begin{minipage}{0.55\linewidth}
     \inputminted[fontsize=\scriptsize,resetmargins]{latex}{example.out}
   \end{minipage} &
   \begin{minipage}{0.35\linewidth}
     \setlength{\parskip}{6pt plus 1pt minus 1pt}%
     \raggedright\scriptsize\input{example.out}
   \end{minipage}
   \end{tabular}}}

\newenvironment{exampletwouptinynoframe}
  {\VerbatimEnvironment
   \begin{VerbatimOut}{example.out}}
  {\end{VerbatimOut}
   \setlength{\parindent}{0pt}
   \begin{tabular}{l|l}
   \begin{minipage}{0.55\linewidth}
     \inputminted[fontsize=\scriptsize,resetmargins]{latex}{example.out}
   \end{minipage} &
   \begin{minipage}{0.35\linewidth}
     \setlength{\parskip}{6pt plus 1pt minus 1pt}%
     \raggedright\scriptsize\input{example.out}
   \end{minipage}
   \end{tabular}}
   
